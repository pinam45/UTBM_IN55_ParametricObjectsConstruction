%!TEX program = xelatex
\documentclass[11pt,class=book]{standalone}
%\usepackage[utf8]{inputenc}
\usepackage[french]{babel}
\usepackage[french]{translator}
\usepackage[T1]{fontenc}
\usepackage{fontspec}
\usepackage[table,svgnames]{xcolor}

\usepackage{pgf}
\usepackage{tikz}

\usetikzlibrary{shapes}
\usetikzlibrary{arrows.meta}
\usetikzlibrary{calc}
\usetikzlibrary{matrix}


\begin{document}
	\pgfmathsetmacro\layerasize{120}%
	\pgfmathsetmacro\layerainitialrotation{18}%
	\pgfmathsetmacro\layerbsize{45}%
	\pgfmathsetmacro\layerbinitialrotation{0}%

	\begin{tikzpicture}[x=1pt,y=1pt,>=Latex]
		\tikzset{triangleface/.style={
			fill=yellow!35
		}}
		\tikzset{squareface/.style={
			fill=orange!35
		}}
		\tikzset{line/.style={
			thick
		}}
		\tikzset{layersline/.style={
			line,
			black
		}}
		\tikzset{nearestline/.style={
			line,
			Green
		}}
		\tikzset{squaresplitline/.style={
			line,
			Red
		}}

		%---------------------------------
		% Inner layer points
		\coordinate (P0) at (\layerbinitialrotation+0:\layerbsize);
		\coordinate (P1) at (\layerbinitialrotation+90:\layerbsize);
		\coordinate (P2) at (\layerbinitialrotation+180:\layerbsize);
		\coordinate (P3) at (\layerbinitialrotation+270:\layerbsize);

		%---------------------------------
		% Outter layer points
		\coordinate (P4) at (\layerainitialrotation+0:\layerasize);
		\coordinate (P5) at (\layerainitialrotation+72:\layerasize);
		\coordinate (P6) at (\layerainitialrotation+144:\layerasize);
		\coordinate (P7) at (\layerainitialrotation+216:\layerasize);
		\coordinate (P8) at (\layerainitialrotation+288:\layerasize);

		%---------------------------------
		% Numbers
		\node[left] at (P0) {0};
		\node[below] at (P1) {1};
		\node[right] at (P2) {2};
		\node[above] at (P3) {3};
		\node[right] at (P4) {4};
		\node[above] at (P5) {5};
		\node[left] at (P6) {6};
		\node[left] at (P7) {7};
		\node[right] at (P8) {8};

		%---------------------------------
		% Lines
		\draw[nearestline] (P4) -- (P0);
		\draw[nearestline] (P5) -- (P1);
		\draw[nearestline] (P6) -- (P2);
		\draw[nearestline] (P7) -- (P3);
		\draw[nearestline] (P8) -- (P3);

		%---------------------------------
		% Layers lines
		\draw[layersline] (P0) -- (P1) -- (P2) -- (P3) -- cycle;
		\draw[layersline] (P4) -- (P5) -- (P6) -- (P7) -- (P8) -- cycle;
	\end{tikzpicture}
\end{document}
