\PassOptionsToPackage{table,svgnames}{xcolor}
\documentclass[article, backcover, french, nodocumentinfo]{upmethodology-document}
\include{packages}
%!TEX root = ./main.tex

%----------------------------------------
% upmethodology commands redefinition
%----------------------------------------

\makeatletter

% Remove 'Initials' column from validators
\renewcommand{\upm@document@addvalidator}[3][]{%
	\global\protected@edef\thevalidatorlist{\thevalidatorlist\protect\Ifnotempty{\thevalidatorlist}{,} \protect\upmmakename{#2}{#3}{~}}

	\global\protected@edef\upm@document@validator@tab@commented{\upm@document@validator@tab@commented \protect\upmmakename{#2}{#3}{~} & 
	& \protect\Ifnotempty{#1}{\protect\href{mailto:#1}{#1}}\\}

	\ifupm@document@validator@tab@hascomment\else
		\global\protected@edef\upm@document@validator@tab{\upm@document@validator@tab \protect\upmmakename{#2}{#3}{~} & 
		\protect\Ifnotempty{#1}{\protect\href{mailto:#1}{#1}}\\}
	\fi
}
\renewcommand{\upm@document@addvalidatorstar}[4][]{%
	\global\protected@edef\thevalidatorlist{\thevalidatorlist\protect\Ifnotempty{\thevalidatorlist}{,} \protect\upmmakename{#2}{#3}{~}}

	\global\let\upm@document@validator@tab\relax

	\global\protected@edef\upm@document@validator@tab@commented{\upm@document@validator@tab@commented \protect\upmmakename{#2}{#3}{~} & 
	#4 & \protect\Ifnotempty{#1}{\protect\href{mailto:#1}{#1}}\\}

	\upm@document@validator@tab@hascommenttrue
}
\renewcommand{\upmdocumentvalidators}[1][\linewidth]{%
	\ifupm@document@validator@tab@hascomment%
		\Ifnotempty{\upm@document@validator@tab@commented}{%
		\noindent\expandafter\begin{mtabular}[#1]{3}{|X|l|c|}%
		\tabulartitle{\upm@lang@document@validators}%
		\tabularheader{\upm@lang@document@names}{\upm@lang@document@comments}{\upm@lang@document@emails}%
		\upm@document@validator@tab@commented
		\hline%
		\expandafter\end{mtabular}\par\vspace{.5cm}}%
	\else%
		\Ifnotempty{\upm@document@validator@tab}{%
		\noindent\expandafter\begin{mtabular}[#1]{2}{|X|c|}%
		\tabulartitle{\upm@lang@document@validators}%
		\tabularheader{\upm@lang@document@names}{\upm@lang@document@emails}%
		\upm@document@validator@tab
		\hline%
		\expandafter\end{mtabular}\par\vspace{.5cm}}%
	\fi%
}

% Remove history from document info page
\renewcommand{\upmdocinfopage}{
	\thispagestyle{plain}
	\upmdocumentsummary\upmdocumentauthors\upmdocumentvalidators\upmdocumentinformedpeople\clearpage%
}

% Decrease space after upmcaution upminfo and upmquestion message boxes
\renewenvironment{upm@highligh@box}[2]{%
	\par
	\vspace{.5cm}
	\begin{tabular}{|p{#1}|}
	\hline
	\begin{window}[0,l,{\mbox{\includegraphics[width=1cm]{#2}}},{}]
}{%
	\end{window}\\ \hline \end{tabular}
	%\vspace{.5cm}
	\par
}

\makeatother

%----------------------------------------
% upmethodology informations
%----------------------------------------

%%% Document Information and Declaration
\declaredocument{Construction d'objet paramétrique}{IN55}{--}

%%% Abstract and Key-words
\setdocabstract[french]{Projet UTBM de l'UV IN55 du semestre de printemps 2018}
\setdockeywords[french]{UTBM, IN55}

%%% Document Authors and Validators
\addauthorvalidator*[julien.barbier@utbm.fr]{Julien}{Barbier}{UTBM/INFO04/I2RV}
\addauthorvalidator*[jerome.boulmier@utbm.fr]{Jérôme}{Boulmier}{UTBM/INFO04/I2RV}
\addauthorvalidator*[maxime.pinard@utbm.fr]{Maxime}{Pinard}{UTBM/INFO04/I2RV}

%%% Informed People
\addinformed*[fabrice.lauri@utbm.fr]{Fabrice}{Lauri}{Professeur de l'UV IN55}

%%% Copyright and Publication Information
\setcopyrighter{Julien Barbier, Jérôme Boulmier et Maxime Pinard}
\setpublisher{l'UTBM}
\setprintingaddress{France}

%%% Version
\incversion{\makedate{\the\day}{\the\month}{\the\year}}{Initial version.}{\upmpublic}

%----------------------------------------
% Other configurations
%----------------------------------------

% Figures folder
\graphicspath{{figures/}}

% Change Front Page Layout
%\setfrontcover{modern} % modern or classic

% Change Illustration Picture
%\setfrontillustration[1.3]{figures/figure}

% Source code formatting
\upmcodelang{cpp} % uml, java or cpp

% Prevent page breaks in paragraphs
\predisplaypenalty=1000
\postdisplaypenalty=1000
\clubpenalty=1000

% Minimal space required in the bottom margin not to move the title on the next page
%\renewcommand{\bottomtitlespace}{.1\textheight}

% Links config, especialy for the table of contents
\hypersetup{
    colorlinks=true,
    linkcolor=black,
    urlcolor=blue,
    linktoc=all
}

% French language config
\frenchbsetup{StandardLayout=true,ReduceListSpacing=false,CompactItemize=false}

%----------------------------------------
% Functions definitions
%----------------------------------------

%Paragraph with line break
\newcommand{\p}[1]{\paragraph{#1\\}}

% Function to print a warning sign
\newcommand{\dangersign}[1][2.5ex]
	{\renewcommand{\stacktype}{L}
		{\scaleto{\stackon[1pt]{\color{red}$\triangle$}{\fontsize{4pt}{4pt}\selectfont !}}{#1}}}

% Definition of some dt/dx/dy shortcuts for integrals
\newcommand{\dt}
{\;\mathrm{d}\,t}

\newcommand{\dx}
{\;\mathrm{d}\,x}

\newcommand{\dy}
{\;\mathrm{d}\,y}

% Definition of \Witem for 'itemize' environment with a warning sign
\newcommand{\Witem}
{\item[\dangersign{}]}

% Definition of a Max function shortcut
\newcommand{\Max}[2][ ]
{\underset{#1}{\text{Max}}\,#2}

% Definition of a Min function shortcut
\newcommand{\Min}[2][ ]
{\underset{#1}{\text{Min}}\,#2}

%----------------------------------------
% Figures
%----------------------------------------

\input{figures/figures_common}


\newcommand{\TODO}[2][ ]{\todo[inline,color=green]{#2}}

\begin{document}
	\thispagestyle{empty}
	\upmdocumentsummary{}
	\upmdocumentauthors{}
	%\upmdocumentvalidators{}
	\upmdocumentinformedpeople{}
	\upmpublicationpage{}
	\newpage{}
	\tableofcontents{}
	%\listoffigures{}
	\newpage{}
	\section{Introduction}
		\paragraph*{}
			Dans le cadre de l'UV IN55 pour le semestre de printemps 2018, la réalisation d'un projet mettant en oeuvre OpenGL était demandée.
			Plusieurs sujets étaient proposés, nous avons choisi de traiter le sujet ``Construction d'un objet paramétrique''.
		\paragraph*{}
			Dans un premier temps, nous allons expliquer ce que nous avons choisi de réaliser.
			Nous présenterons ensuite l'installation et l'utilisation du programme.
			Puis nous finirons par la conception, qui expliquera les technologies utilisées ainsi que l'algorithme que nous avons conçu pour générer l'objet paramétrique.

	\section{Sujet choisi}
		\paragraph*{}
			Dans cette section, nous allons détailler la phase d'étude qui a précédé le développement de notre solution.
			Nous avons choisi de réaliser un programme permettant la génération d'un grand nombre de polygones.
		\paragraph*{}
			Pour ce faire, nous avons choisi une approche a base de polygone régulier convexes, c'est-à-dire des polygones dont tous les cotés ont la même longueur et tous les angles la même mesure. Cette propriété les rend facilement constructibles à partir de l'équation d'un cercle.
		\paragraph*{}
			En plaçant ces polygones de façon espacée sur l'axe Z et en les reliant avec des faces triangulaires il nous est possible de générer de nombreux polyèdres.
		\paragraph*{}
			Le générateur que nous avons réalisé dispose des paramètres suivants:
			\begin{itemize}
				\item Choix du nombre de polygone, appelé \textit{layers}
				\item Choix de la taille des polygones
				\item Choix du nombre de points de chaque polygone
				\item Possibilité de tourner chaque polygone
				\item Choix de la couleur de chaque polygone
			\end{itemize}

	\section{Installation et utilisation}
		\subsection{Installation}
			\paragraph*{}
				Pour ouvrir le projet sur Visual Studio 2017:
				\begin{itemize}
					\item Installer CMake disponible à cette addresse:\\
						\url{https://cmake.org/download/}
					\item Lancer CMake (cmake-gui)
					\item Sélectionner la racine du dossier de sources
					\item Sélectionner le dossier de build \\
						\includegraphics[width=0.6\textwidth]{tuto1}
					\item Appuyer sur "Configure"
					\item Sélectionner "Visual Studio 15 2017 Win64" en tant que "générateur" \\
						\includegraphics[width=0.75\textwidth]{tuto2}
					\item Appuyer sur "Finish"
					\item Une fois la configuration générée, appuyer sur "Generate"
					\item Appuyer sur "Open Project"
					\item Sélectionner "ParametricObjectsConstruction" comme projet de démarrage (dans l'explorateur de solution) en appuyant sur "Définir comme projet de démarrage" \\
						\includegraphics[width=0.5\textwidth]{tuto3}
				\end{itemize}

		\subsection{Utilisation}
			\paragraph*{}
				Afin de modifier l'objet paramétrique, on pourra utiliser le menu disponible sur la gauche du programme~: \\
				\begin{minipage}[c]{0.4\textwidth}
					\begin{figure}[H]
						\centering%
						\includegraphics[width=0.8\textwidth]{menu}%
						\caption{Menu de configuration}%
						\label{fig:menu}%
					\end{figure}
				\end{minipage}
				\begin{minipage}[c]{0.59\textwidth}
					\begin{figure}[H]
						Dans ce programme les objets seront construits en utilisant des layers. Le bouton \fbox{\texttt{+}} permet d'insérer un nouveau layer et les boutons \fbox{$\blacktriangle$} et \fbox{$\blacktriangledown$} permettent de déplacer le layer.\\
						\hfill \\
						Chaque Layer correspond à un polygone configurable, le bouton \fbox{\texttt{Randomize}} permet de générer des valeurs de configuration aléatoire pour le layer.\\
						\hfill \\
						Les configurations disponibles sont les suivantes:
						\begin{itemize}
							\item \texttt{points}: nombre de points du polygone
							\item \texttt{radius}: rayon du polygone (distance des points au centre)
							\item \texttt{rotation}: rotation du polygone (sur l'axe Z)
							\item \texttt{distance}: distance du layer avec le précédent
							\item \texttt{color}: couleur du polygone
						\end{itemize}
						\hfill \\
						Enfin, la croix sur la barre qui contient le numéro du layer permet de le supprimer.
					\end{figure}
				\end{minipage}
	\section{Conception et technologies utilisées}
		\paragraph*{}
			Dans notre projet, nous avons choisi de réaliser un mini-wrapper autour d'OpenGl afin de bénéficier des fonctionnalités du C++.
			De plus, afin de réaliser l'interface utilisateur, nous avons utilisé "ImGUI".
		\paragraph*{}
			ImGUI est une librairie générant des "vertex buffers".
			Pour pouvoir afficher les interfaces réalisées avec ImGUI, nous avons dû réaliser une implémentation permettant d'afficher les composants générés.

			\begin{figure}[H]
				\centering%
				\includegraphics[width=\textwidth]{uml}
				\caption{Diagramme de classes du projet}
				\label{fig:uml}
			\end{figure}

		\paragraph*{}
			Notre wrapper est principalement axé autour de la classe Window qui permet de gérer la fenêtre ainsi que les events reçus par celle-ci.
			Ces events sont récupérés grâce à la fonction "pollEvent" qui remplira la structure Event. Cette fonction retournera vrai tant qu'il reste des events à traiter.
		\paragraph*{}
			Une fois la fenêtre initialisée, nous pourrons utiliser les classes "Shader" et "ShaderProgram" afin de charger et d'utiliser les shaders.
			La classe ParametricObject permettra, elle, de générer l'objet paramétrique.

	\section{Construction de l'objet paramétrique}
	\paragraph{} L'objet paramétrique implémenté dans ce projet permet la réalisation d'un ensemble d'objet différent. L'objet se construit autour de layers qui possèdent différents paramètres comme la taille du layer, le nombre de points, la rotation et la couleur.
	\paragraph{} La construction de l'objet paramétrique se passe en deux parties. La première partie est la génération du tableau de vertices, la deuxième la génération du tableau d'indices. La première partie va permettre de générer les sommets de l'objets et la deuxième partie va permettre la génération des faces des différents layers de l'objet.
	\subsection{Calcul des vertices de l'objet}
	\subsubsection{Calcul des vertices pour un layer}\label{a}
		\paragraph{} Les layers sont construits comme des polygones réguliers convexes.
		\paragraph{}Les variables nécessaires pour l'application de l'algorithme sont : 
		\begin{itemize}
		 	\item $i$ : le numéros du layer courant
		 	\item $h_{progressif}$ : la hauteur de l'objet au layer $i$
		 	\item $h_{objet}$ : la taille de l'objet
		 	\item $\theta$ : l'angle entre les vertices du layer
		 	\item $r$ : le rayon du layer
		 	\item $n$ : le nombre de vertices de la face.
		 	\item $rot$ : la rotation du layer.
		\end{itemize} L'algorithme permettant la construction du layer se décompose de la manière suivante :
		\begin{itemize}
			\item Calculer la coordonnée $z$ en appliquant la formule suivante : $z = \frac{h_{objet}}{2} - h_{progressif}$.
			\item Calculer l'angle en radian entre deux vertices avec la formule suivante : $\theta = \frac{2\pi}{n}+rot$.
			\item Pour chaque vertice $j$, on applique : 
			\begin{itemize}
				\item Calculer la coordonnée $x$ en appliquant la formule $x = cos(\theta*j)*r$.
				\item Calculer la coordonnée $y$ en appliquant la formule $y = sin(\theta*j)*r$.
			\end{itemize}			
		\end{itemize}
	\subsubsection{Calcul des vertices de l'objet via la fonction de calcul des vertices d'une face}
	\paragraph{} Pour calculer les vertices de l'objet, il suffit d'itérer sur les layers de l'objet en appelant la fonction de création de vertices pour un layer (cf section \ref{a}).
	\subsection{Calcul des faces de l'objet avec le tableau d'indices}
	\paragraph{}Pour pouvoir calculer les faces de l'objet, il a été décidé d'utiliser un tableau d'indice. Ce tableau permet d'éviter de répéter le vertice dans le tableau de vertices et ainsi un meilleur déverminage (debug en anglais) lors de l'exécution du programme. Le calcul des faces de l'objet a été découpées en 2 parties :
	\begin{itemize}
		\item Le calcul des faces du layer
		\item Le calcul des faces de liaison entre les layers
	\end{itemize}
	\paragraph{} La primitive \texttt{GL\_TRIANGLES} a été choisi pour le rendu de l'objet paramétrique. Cette primitive permet la création des faces de l'objet de manière souple ce qui est nécessaires pour les algorithmes choisis.
	\subsubsection{Calcul des faces d'un layer}
	\paragraph{} Pour construire les faces du layer, on procède par étapes : 
	\begin{enumerate}
		\item On prend les vertices 3 par 3 et on calcul les triangles obtenus. Dans la figure \ref{fig:face_1} on prend les ensembles (0-1-2), (2-3-4) et (4-5-6).
		\item On prend ensuite le premier point du layer puis deux vertices "paires" consécutifs. Par exemple dans la figure \ref{fig:face_2}, on obtient : (0-2-4), (0-4-6).
	\end{enumerate}
	On obtient les deux figures \ref{fig:face_1} et \ref{fig:face_2}.
	\begin{figure}[H]
			\centering%
			\resizebox{0.32\textwidth}{!}{\import{figures/}{face_1_1.tex}}%
			\resizebox{0.32\textwidth}{!}{\import{figures/}{face_1_2.tex}}%
			\resizebox{0.32\textwidth}{!}{\import{figures/}{face_1_3.tex}}%
			\caption{Construction d'une face pair}%
			\label{fig:face_1}%
	\end{figure}
	\begin{figure}[H]
			\centering%
			\resizebox{0.32\textwidth}{!}{\import{figures/}{face_2_1.tex}}%
			\resizebox{0.32\textwidth}{!}{\import{figures/}{face_2_2.tex}}%
			\resizebox{0.32\textwidth}{!}{\import{figures/}{face_2_3.tex}}%
			\caption{Construction d'une face impaire}%
			\label{fig:face_2}%
	\end{figure}
	\subsubsection{Liaison des faces dans l'objet paramétrique}
		\paragraph{} Pour pouvoir terminer notre objet paramétrique, nous devions lier les faces entre elles. On voulait pour cet objet pouvoir lier des faces qui ne possédaient pas forcément le même nombre de vertices. Plusieurs algorithmes ont été testés mais seul le dernier algorithme, qui présente le meilleur résultat, sera expliqué.
		
		Prenons comme exemple la figure \ref{fig:layers_1_empty}. On a une face qui possède 4 vertices et l'autre face qui possède 5 vertices. On nomme la face ayant le moins de vertices A et l'autre B. On relie chaque vertice de la face B avec le vertice le plus proche de la face A. Ainsi le vertice 4 a comme vertice le plus proche 0 dans la face A.
		
		Après avoir itérer sur tous les vertices, on obtient la figure \ref{fig:layers_2_nearest_lines}.
		
		\par\noindent\begin{minipage}[t]{\textwidth}
			\centering
			\begin{minipage}[t]{0.49\textwidth}
				\begin{figure}[H]
					\centering%
					\resizebox{\textwidth}{!}{\import{figures/}{layers_1_empty.tex}}%
					\caption{Faces non liés}%
					\label{fig:layers_1_empty}%
				\end{figure}
			\end{minipage}
			\begin{minipage}[t]{0.49\textwidth}
				\begin{figure}[H]
					\centering%
					\resizebox{\textwidth}{!}{\import{figures/}{layers_2_nearest_lines.tex}}%
					\caption{Liaison avec les points les plus proches}%
					\label{fig:layers_2_nearest_lines}%
				\end{figure}
			\end{minipage}
		\end{minipage}
		\paragraph{} On va ensuite remplir les faces triangulaires. Pour ce faire, on va regarder si on a des vertices de la face B qui sont rattachées au même vertice de la face A. Par exemple dans la figure \ref{fig:layers_2_nearest_lines}, on a le vertice 7 et le vertice 8 qui sont liés au point 3. On a donc ici une face triangulaire qui peut être calculé par OpenGL avec la primitive GL\_TRIANGLES sous la forme (7-3-8). On obtient donc la figure \ref{fig:layers_3_triangles_quadrilaterals}.
		
		Pour finir, on va calculer les quadrilatères. Comme on utilise la primitive GL\_TRIANGLES, il faut diviser ces faces en deux triangles. Pour pouvoir diviser ces faces, on a décidé de comparer la longueur des diagonales de ces faces. On voit par exemple que la diagonale 0-5 est plus longue que la diagonale 1-4. Ainsi on va diviser la face quadrilatère avec la diagonale la plus courte et on va remplir les faces triangulaires obtenues. On obtient donc la figure \ref{fig:layers_4_quadrilaterals_diagonals}. On a fini de lier les faces.
		\par\noindent\begin{minipage}[t]{\textwidth}
			\centering
			\begin{minipage}[t]{0.49\textwidth}
				\begin{figure}[H]
					\centering%
					\resizebox{\textwidth}{!}{\import{figures/}{layers_3_triangles_quadrilaterals.tex}}%
					\caption{Délimitation de triangles et quadrilatères}
					\label{fig:layers_3_triangles_quadrilaterals}%
				\end{figure}
			\end{minipage}
			\begin{minipage}[t]{0.49\textwidth}
				\begin{figure}[H]
					\centering%
					\resizebox{\textwidth}{!}{\import{figures/}{layers_4_quadrilaterals_diagonals.tex}}%
					\caption{Coupure des quadrilatères sur la diagonale la plus courte}%
					\label{fig:layers_4_quadrilaterals_diagonals}%
				\end{figure}
			\end{minipage}
		\end{minipage}
		
		Après implémentation dans le projet de cette algorithme, on obtient un objet paramétrique complet et sans faces creuses comme on peux voir dans les exemples \ref{fig:view_1_1} et \ref{fig:view_1_2}.
		\par\noindent\begin{minipage}[t]{\textwidth}
			\centering
			\begin{minipage}[t]{0.49\textwidth}
				\begin{figure}[H]
					\centering%
					\includegraphics[width=1\textwidth]{view_1_1}%
					\caption{Exemple: vu de coté}%
					\label{fig:view_1_1}%
				\end{figure}
			\end{minipage}
			\begin{minipage}[t]{0.49\textwidth}
				\begin{figure}[H]
					\centering%
					\includegraphics[width=1\textwidth]{view_1_2}%
					\caption{Exemple: vu du dessus}%
					\label{fig:view_1_2}%
				\end{figure}
			\end{minipage}
		\end{minipage}
\end{document}
